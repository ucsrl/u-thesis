%%%%%%%%%%%%%%%%%%%%%%%%%%%%%%%%%%%%%%%%%%%%%%%%%%%%%%%%%%%%%%%
%% OXFORD THESIS TEMPLATE

% Use this template to produce a standard thesis that meets the Oxford University requirements for DPhil submission
%
% Originally by Keith A. Gillow (gillow@maths.ox.ac.uk), 1997
% Modified by Sam Evans (sam@samuelevansresearch.org), 2007
% Modified by John McManigle (john@oxfordechoes.com), 2015
%
% This version Copyright (c) 2015-2023 John McManigle
%
% Broad permissions are granted to use, modify, and distribute this software
% as specified in the MIT License included in this distribution's LICENSE file.
%

% I've (John) tried to comment this file extensively, so read through it to see how to use the various options.  Remember
% that in LaTeX, any line starting with a % is NOT executed.  Several places below, you have a choice of which line to use
% out of multiple options (eg draft vs final, for PDF vs for binding, etc.)  When you pick one, add a % to the beginning of
% the lines you don't want.


%%%%% CHOOSE PAGE LAYOUT
% The most common choices should be below.  You can also do other things, like replacing "a4paper" with "letterpaper", etc.

% This one will format for two-sided binding (ie left and right pages have mirror margins; blank pages inserted where needed):
\documentclass[a4paper,twoside]{µthesis}
% This one will format for one-sided binding (ie left margin > right margin; no extra blank pages):
%\documentclass[a4paper]{µthesis}
% This one will format for PDF output (ie equal margins, no extra blank pages):
%\documentclass[a4paper,nobind]{µthesis} 

% Use Noto fonts - comment out to use default LaTeX fonts
\usepackage{fontspec}
\usepackage{noto-mono}
\usepackage{notomath}


%%%%% SELECT YOUR DRAFT OPTIONS
% Three options going on here; use in any combination.  But remember to turn the first two off before
% generating a PDF to send to the printer!

% This adds a "DRAFT" footer to every normal page.  (The first page of each chapter is not a "normal" page.)
\fancyfoot[C]{\emph{DRAFT Printed on \today}}  

% This highlights (in blue) corrections marked with (for words) \mccorrect{blah} or (for whole
% paragraphs) \begin{mccorrection} . . . \end{mccorrection}.  This can be useful for sending a PDF of
% your corrected thesis to your examiners for review.  Turn it off, and the blue disappears.
\correctionstrue


%%%%% BIBLIOGRAPHY SETUP

\usepackage[style=ieee, sortcites=true, backend=biber, doi=true, isbn=false,
            maxnames=1, minnames=1, maxbibnames=99, citestyle=numeric-comp]{biblatex}
\newcommand*{\bibtitle}{References}

% Environment for the Puplications page
\defbibenvironment{nolabelspaced}
  {\list{}{\leftmargin=0pt
           \itemindent=0pt
           \labelwidth=0pt
           \labelsep=0pt
           \setlength{\itemsep}{4ex} % space between items
           \setlength{\parsep}{0pt}
           }
   \renewcommand*{\makelabel}[1]{}}%
  {\endlist}
  {\item}

% You can use the following snippet to select your publications, just replace "Patton" with your name.
% Alternatively, you can set "keywords={own}" for bib entries your dissertation is based on.
\DeclareSourcemap{
  \maps[datatype=bibtex]{
    \map{
      \step[fieldsource=author, match=Patton, final]
      \step[fieldset=keywords, fieldvalue=own]
    }
  }
}

% This makes the bibliography use a slightly smaller font.
\renewcommand*{\bibfont}{\small}

% Suppress warnings...
\DeclareLanguageMapping{latin}{english}
\BiblatexSplitbibDefernumbersWarningOff

% Change this to the name of your .bib file (usually exported from a citation manager like Zotero or EndNote).
\addbibresource{references.bib}


% Uncomment this if you want equation numbers per section (2.3.12), instead of per chapter (2.18):
%\numberwithin{equation}{subsection}




%%%%% THESIS / TITLE PAGE INFORMATION
\title{Suitably impressive thesis title}
\author{Your Name}
\department{Informatik}
% Your full degree name.
\degree{Doktors (Dr.)}
% Your examiners/supervisors.
\firstexaminer{Prof. A}
\secondexaminer{Prof. B}
% You don't need these dates for your initial submission, they must be set for the final submission.
% \submissiondate{01.01.2026}
% \acceptancedate{01.02.2026}
% \examinationdate{01.03.2026}


%%%%% YOUR OWN PERSONAL MACROS
% This is a good place to dump your own LaTeX macros as they come up.

% To make text superscripts shortcuts
	\renewcommand{\th}{\textsuperscript{th}} % ex: I won 4\th place
	\newcommand{\nd}{\textsuperscript{nd}}
	\renewcommand{\st}{\textsuperscript{st}}
	\newcommand{\rd}{\textsuperscript{rd}}

%%%%% BOXED ENVIRONMENTS
% Some suggestions for boxed environments, see a usage example in Chapter 2.
\newboxenv{excursus}{Excursus}{Excursuses}{unibwgraylight}
\newboxenv{definition}{Definition}{Definitions}{unibwbluelight}
\newboxenv{observation}{Observation}{Observations}{unibworangelight}

\DeclareAcronymWithTooltip{pet}{PET}{positron emission tomography}
\DeclareAcronymWithTooltip{spect}{SPECT}{single-photon emission computed tomography}

% or use \DeclareAcronym if you don't want tooltips

%%%%% THE ACTUAL DOCUMENT STARTS HERE
\begin{document}



%%%%% CHOOSE YOUR LINE SPACING HERE
\setlength{\textbaselineskip}{15pt plus2pt minus1pt}

% You can set the spacing here for the roman-numbered pages (acknowledgements, table of contents, etc.)
\setlength{\frontmatterbaselineskip}{15pt plus2pt minus1pt}

% Leave this line alone; it gets things started for the real document.
\setlength{\baselineskip}{\textbaselineskip}


%%%%% CHOOSE YOUR SECTION NUMBERING DEPTH HERE
% You have two choices.  First, how far down are sections numbered?  (Below that, they're named but
% don't get numbers.)  Second, what level of section appears in the table of contents?  These don't have
% to match: you can have numbered sections that don't show up in the ToC, or unnumbered sections that
% do.  Throughout, 0 = chapter; 1 = section; 2 = subsection; 3 = subsubsection, 4 = paragraph...

% The level that gets a number:
\setcounter{secnumdepth}{2}
% The level that shows up in the ToC:
\setcounter{tocdepth}{2}


% JEM: Pages are roman numbered from here, though page numbers are invisible until ToC.  This is in
% keeping with most typesetting conventions.
\begin{romanpages}

% JEM: By default, this template uses the traditional Oxford "Belt Crest". Un-comment the following
% line to use the newer, "Blue Square" logo:
% \renewcommand{\crest}{{\includegraphics[width=4.2cm, height=4.2cm]{figures/newlogo.pdf}}}

% Title page is created here
\maketitle

%%%%% DEDICATION -- If you'd like one, un-comment the following.
%\begin{dedication}
%This thesis is dedicated to\\
%someone\\
%for some special reason\\
%\end{dedication}

%%%%% ACKNOWLEDGEMENTS -- Nothing to do here except comment out if you don't want it.
\begin{acknowledgements}
 	\subsection*{Personal}

This is where you thank your advisor, colleagues, and family and friends.

Lorem ipsum dolor sit amet, consectetur adipiscing elit. Vestibulum feugiat et est at accumsan. Praesent sed elit mattis, congue mi sed, porta ipsum. In non ullamcorper lacus. Quisque volutpat tempus ligula ac ultricies. Nam sed erat feugiat, elementum dolor sed, elementum neque. Aliquam eu iaculis est, a sollicitudin augue. Cras id lorem vel purus posuere tempor. Proin tincidunt, sapien non dictum aliquam, ex odio ornare mauris, ultrices viverra nisi magna in lacus. Fusce aliquet molestie massa, ut fringilla purus rutrum consectetur. Nam non nunc tincidunt, rutrum dui sit amet, ornare nunc. Donec cursus tortor vel odio molestie dignissim. Vivamus id mi erat. Duis porttitor diam tempor rutrum porttitor. Lorem ipsum dolor sit amet, consectetur adipiscing elit. Sed condimentum venenatis consectetur. Lorem ipsum dolor sit amet, consectetur adipiscing elit.

Aenean sit amet lectus nec tellus viverra ultrices vitae commodo nunc. Mauris at maximus arcu. Aliquam varius congue orci et ultrices. In non ipsum vel est scelerisque efficitur in at augue. Nullam rhoncus orci velit. Duis ultricies accumsan feugiat. Etiam consectetur ornare velit et eleifend.

Suspendisse sed enim lacinia, pharetra neque ac, ultricies urna. Phasellus sit amet cursus purus. Quisque non odio libero. Etiam iaculis odio a ex volutpat, eget pulvinar augue mollis. Mauris nibh lorem, mollis quis semper quis, consequat nec metus. Etiam dolor mi, cursus a ipsum aliquam, eleifend venenatis ipsum. Maecenas tempus, nibh eget scelerisque feugiat, leo nibh lobortis diam, id laoreet purus dolor eu mauris. Pellentesque habitant morbi tristique senectus et netus et malesuada fames ac turpis egestas. Nulla eget tortor eu arcu sagittis euismod fermentum id neque. In sit amet justo ligula. Donec rutrum ex a aliquet egestas.

\subsection*{Institutional}

If you want to separate out your thanks for funding and institutional support, I don't think there's any rule against it.  Of course, you could also just remove the subsections and do one big traditional acknowledgement section.

Lorem ipsum dolor sit amet, consectetur adipiscing elit. Ut luctus tempor ex at pretium. Sed varius, mauris at dapibus lobortis, elit purus tempor neque, facilisis sollicitudin felis nunc a urna. Morbi mattis ante non augue blandit pulvinar. Quisque nec euismod mauris. Nulla et tellus eu nibh auctor malesuada quis imperdiet quam. Sed eget tincidunt velit. Cras molestie sem ipsum, at faucibus quam mattis vel. Quisque vel placerat orci, id tempor urna. Vivamus mollis, neque in aliquam consequat, dui sem volutpat lorem, sit amet tempor ipsum felis eget ante. Integer lacinia nulla vitae felis vulputate, at tincidunt ligula maximus. Aenean venenatis dolor ante, euismod ultrices nibh mollis ac. Ut malesuada aliquam urna, ac interdum magna malesuada posuere.
\end{acknowledgements}

%%%%% ABSTRACT -- Nothing to do here except comment out if you don't want it.
\renewcommand{\abstractname}{Kurzfassung}
\begin{abstract}
    Lorem ipsum dolor sit amet, consectetur adipiscing elit. Pellentesque sit amet nibh volutpat, scelerisque nibh a, vehicula neque. Integer placerat nulla massa, et vestibulum velit dignissim id. Ut eget nisi elementum, consectetur nibh in, condimentum velit. Quisque sodales dui ut tempus mattis. Duis malesuada arcu at ligula egestas egestas. Phasellus interdum odio at sapien fringilla scelerisque. Mauris sagittis eleifend sapien, sit amet laoreet felis mollis quis. Pellentesque dui ante, finibus eget blandit sit amet, tincidunt eu neque. Vivamus rutrum dapibus ligula, ut imperdiet lectus tincidunt ac. Pellentesque ac lorem sed diam egestas lobortis.

Suspendisse leo purus, efficitur mattis urna a, maximus molestie nisl. Aenean porta semper tortor a vestibulum. Suspendisse viverra facilisis lorem, non pretium erat lacinia a. Vestibulum tempus, quam vitae placerat porta, magna risus euismod purus, in viverra lorem dui at metus. Sed ac sollicitudin nunc. In maximus ipsum nunc, placerat maximus tortor gravida varius. Suspendisse pretium, lorem at porttitor rhoncus, nulla urna condimentum tortor, sed suscipit nisi metus ac risus.

Aenean sit amet enim quis lorem tristique commodo vitae ut lorem. Duis vel tincidunt lacus. Sed massa velit, lacinia sed posuere vitae, malesuada vel ante. Praesent a rhoncus leo. Etiam sed rutrum enim. Pellentesque lobortis elementum augue, at suscipit justo malesuada at. Lorem ipsum dolor sit amet, consectetur adipiscing elit. Praesent rhoncus convallis ex. Etiam commodo nunc ex, non consequat diam consectetur ut. Pellentesque vitae est nec enim interdum dapibus. Donec dapibus purus ipsum, eget tincidunt ex gravida eget. Donec luctus nisi eu fringilla mollis. Donec eget lobortis diam.

Suspendisse finibus placerat dolor. Etiam ornare elementum ex ut vehicula. Donec accumsan mattis erat. Quisque cursus fringilla diam, eget placerat neque bibendum eu. Ut faucibus dui vitae dolor porta, at elementum ipsum semper. Sed ultrices dui non arcu pellentesque placerat. Etiam posuere malesuada turpis, nec malesuada tellus malesuada.
\end{abstract}
\cleardoublepage

\renewcommand{\abstractname}{Abstract}
\begin{abstract}
    Lorem ipsum dolor sit amet, consectetur adipiscing elit. Pellentesque sit amet nibh volutpat, scelerisque nibh a, vehicula neque. Integer placerat nulla massa, et vestibulum velit dignissim id. Ut eget nisi elementum, consectetur nibh in, condimentum velit. Quisque sodales dui ut tempus mattis. Duis malesuada arcu at ligula egestas egestas. Phasellus interdum odio at sapien fringilla scelerisque. Mauris sagittis eleifend sapien, sit amet laoreet felis mollis quis. Pellentesque dui ante, finibus eget blandit sit amet, tincidunt eu neque. Vivamus rutrum dapibus ligula, ut imperdiet lectus tincidunt ac. Pellentesque ac lorem sed diam egestas lobortis.

Suspendisse leo purus, efficitur mattis urna a, maximus molestie nisl. Aenean porta semper tortor a vestibulum. Suspendisse viverra facilisis lorem, non pretium erat lacinia a. Vestibulum tempus, quam vitae placerat porta, magna risus euismod purus, in viverra lorem dui at metus. Sed ac sollicitudin nunc. In maximus ipsum nunc, placerat maximus tortor gravida varius. Suspendisse pretium, lorem at porttitor rhoncus, nulla urna condimentum tortor, sed suscipit nisi metus ac risus.

Aenean sit amet enim quis lorem tristique commodo vitae ut lorem. Duis vel tincidunt lacus. Sed massa velit, lacinia sed posuere vitae, malesuada vel ante. Praesent a rhoncus leo. Etiam sed rutrum enim. Pellentesque lobortis elementum augue, at suscipit justo malesuada at. Lorem ipsum dolor sit amet, consectetur adipiscing elit. Praesent rhoncus convallis ex. Etiam commodo nunc ex, non consequat diam consectetur ut. Pellentesque vitae est nec enim interdum dapibus. Donec dapibus purus ipsum, eget tincidunt ex gravida eget. Donec luctus nisi eu fringilla mollis. Donec eget lobortis diam.

Suspendisse finibus placerat dolor. Etiam ornare elementum ex ut vehicula. Donec accumsan mattis erat. Quisque cursus fringilla diam, eget placerat neque bibendum eu. Ut faucibus dui vitae dolor porta, at elementum ipsum semper. Sed ultrices dui non arcu pellentesque placerat. Etiam posuere malesuada turpis, nec malesuada tellus malesuada.
\end{abstract}
\cleardoublepage

%%%%% MINI TABLES
% This lays the groundwork for per-chapter, mini tables of contents.  Comment the following line
% (and remove \minitoc from the chapter files) if you don't want this.  Un-comment either of the
% next two lines if you want a per-chapter list of figures or tables.
\dominitoc % include a mini table of contents
%\dominilof  % include a mini list of figures
%\dominilot  % include a mini list of tables

% This aligns the bottom of the text of each page.  It generally makes things look better.
\flushbottom

% This is where the whole-document ToC appears:
\tableofcontents

\listof{excursus}
	\mtcaddchapter
% \mtcaddchapter is needed when adding a non-chapter (but chapter-like) entity to avoid confusing minitoc

\listoffigures
	\mtcaddchapter

% Uncomment to generate a list of tables:
% \listoftables
% 	\mtcaddchapter

%%%%% LIST OF ABBREVIATIONS
% This example includes a list of abbreviations.  Look at preface/abbreviations.tex to see how that file is formatted.
\printacronyms[name=List of Abbreviations]
\mtcaddchapter[List of Abbreviations]

%%%%% LIST OF PUBLICATIONS
% Generates a list of all publications in references.bib with the keyword "own".
\chapter*{Publications}
\mtcaddchapter[Publications] 
This dissertation is based on the following peer-reviewed publications, ordered by recency:
\vspace{2ex}
\newrefcontext[sorting=ddnt] % most recent first
\printbibliography[keyword=own, heading=none, env=nolabelspaced]

% The Roman pages, like the Roman Empire, must come to its inevitable close.
\end{romanpages}


%%%%% CHAPTERS
% Add or remove any chapters you'd like here, by file name (excluding '.tex'):
\flushbottom
\begin{savequote}[8cm]
\textlatin{Neque porro quisquam est qui dolorem ipsum quia dolor sit amet, consectetur, adipisci velit...}

There is no one who loves pain itself, who seeks after it and wants to have it, simply because it is pain...
  \qauthor{--- Cicero's \textit{de Finibus Bonorum et Malorum}}
\end{savequote}

\chapter{\label{ch:1-intro}Introduction} 

\minitoc

\section{Motivation}

The rapid advance of minimally-invasive cardiac procedures promises improvements in patient safety, procedure efficacy, and access to treatment.  While percutaneous coronary intervention (PCI) has become routine and highly effective \cite{bravata_systematic_2007}, catheter procedures in areas such as electrophysiology (EP) and valve replacement are still coming of age.  This progress is driven by demographics and the improvement in general cardiac care, as patients surviving initial cardiac events go on to require treatment for sequelae \cite{foot_demographics_2000}.  The growing need for advanced treatment is being answered by developments in catheter technology and procedures.  These tools are continually advancing to access and manipulate an ever-broader range of anatomy \cite{sousa_new_2005}.

Lorem ipsum dolor sit amet, consectetur adipiscing elit. Maecenas sagittis dolor at nulla feugiat, vitae iaculis est rutrum. Mauris eu sem eros. Sed id faucibus urna. In egestas eros et sapien egestas imperdiet. In hac habitasse platea dictumst. Phasellus vitae varius tortor. Mauris nec sollicitudin enim. Suspendisse molestie leo nec mauris molestie, nec imperdiet magna vehicula. Phasellus sodales tortor dui, a lacinia turpis congue at. Pellentesque mattis dui non libero commodo, at accumsan ex ultrices. Integer eget ex eget dui cursus euismod et accumsan felis. Nullam laoreet sodales dui, ut finibus elit varius a. Sed elementum orci quis libero ullamcorper, eget egestas enim convallis. Sed nibh libero, tincidunt ultricies nibh quis, lobortis placerat mauris. Maecenas at laoreet risus, nec dictum libero. Donec accumsan, orci eu tempus mattis, nisl arcu auctor turpis, ac sollicitudin justo orci nec nulla.

Nam eget sem sed ligula vehicula iaculis. In non arcu a nisl interdum gravida. Nam egestas erat non turpis sagittis vestibulum. Praesent est metus, facilisis eu commodo sed, sagittis et est. Duis scelerisque luctus erat, elementum pulvinar felis bibendum a. Morbi hendrerit rhoncus consectetur. Vestibulum nec odio finibus, blandit turpis eget, dignissim orci. Curabitur eu ligula auctor, porttitor nulla non, maximus turpis. Nunc sed quam at est varius interdum eu vitae odio. Vestibulum egestas dapibus nulla sit amet fermentum.

Vestibulum ut neque urna. Ut nec odio lobortis, ultricies nulla quis, ultricies tellus. Nam ac iaculis sapien. Vivamus vitae risus id tortor interdum pellentesque. Quisque lorem lectus, sagittis vel metus et, sagittis finibus justo. Curabitur pulvinar odio tellus, eu vehicula est dictum eget. Morbi sed justo justo. Vivamus enim nibh, facilisis pretium luctus quis, ullamcorper quis ipsum. Pellentesque a mi a elit euismod malesuada.

Vestibulum interdum est vel orci tincidunt auctor. Nunc tristique nulla nec blandit fermentum. Maecenas id libero ut justo dictum sodales. Nullam justo sapien, dignissim vel enim at, porta pharetra metus. Integer euismod quam eget ligula gravida euismod. Pellentesque commodo, quam sit amet bibendum tempor, nisi odio varius mauris, et accumsan justo ex sed nunc. Cras bibendum nibh ac dolor volutpat, non elementum orci pulvinar. Maecenas et porttitor nulla. Suspendisse sapien massa, dapibus at blandit et, rhoncus suscipit velit. Fusce molestie, velit eget sagittis suscipit, est libero aliquam libero, in iaculis mi tellus ac nunc.

\section{Contribution}

Sed in rhoncus lectus. Mauris vulputate purus non malesuada pulvinar. Curabitur ullamcorper hendrerit elit, id vulputate libero sagittis vel. Pellentesque ac faucibus est. Class aptent taciti sociosqu ad litora torquent per conubia nostra, per inceptos himenaeos. Integer venenatis, nisl eleifend pellentesque consequat, sem tortor malesuada ante, ut tincidunt elit tortor sit amet nunc.

Cras vehicula ipsum sit amet dui rutrum ultrices. Integer eu eleifend odio. Praesent tempor, libero id ullamcorper euismod, lectus diam lobortis mauris, id venenatis arcu sem vitae purus. Pellentesque luctus tristique metus quis mollis. Praesent ullamcorper neque velit, sed iaculis est convallis sit amet. Quisque nec massa ut magna lobortis imperdiet. Quisque rhoncus purus eget mollis aliquet. Donec vehicula viverra nisl, sed posuere turpis vulputate non. Donec malesuada, eros id interdum volutpat, ipsum orci luctus quam, non pulvinar urna ipsum eget purus. Nam hendrerit condimentum tristique.

Proin metus velit, tempor at fringilla non, dictum eu felis. Proin volutpat enim ut fermentum aliquam. Nam dictum nisi eu nisl viverra fermentum. Pellentesque tristique arcu non orci congue faucibus. Fusce sit amet nisl fringilla, feugiat turpis vitae, eleifend ante. Suspendisse elementum, lectus non pulvinar bibendum, lectus massa faucibus turpis, vitae porta risus sem quis metus. Maecenas id sapien et dui lobortis imperdiet nec eu mi. Quisque porttitor tincidunt nisi, eget sagittis orci. Nunc mattis erat malesuada facilisis viverra. Maecenas sodales iaculis nisi vel tincidunt. Morbi aliquet nibh ac facilisis consectetur. In ultrices libero quis massa porttitor cursus. Quisque suscipit ac tortor eget aliquet. Ut eget lacus vel orci viverra maximus at at purus.

Nam massa neque, varius nec suscipit id, cursus ac mi. Cum sociis natoque penatibus et magnis dis parturient montes, nascetur ridiculus mus. In hac habitasse platea dictumst. Vivamus facilisis nunc quis dictum consectetur. Sed congue sed magna non auctor. Vestibulum accumsan sit amet erat non congue. Sed at condimentum mi, sed scelerisque urna. Etiam tristique pulvinar rutrum. Donec semper nulla vitae rutrum semper. Maecenas ultrices nibh at orci sodales tincidunt sit amet vitae arcu. Curabitur interdum tincidunt ipsum, nec tincidunt nunc dapibus in. Nunc sit amet elementum massa, ut ornare lacus. Vivamus convallis fringilla erat, non suscipit sapien convallis eu. Nunc viverra lectus sit amet turpis viverra, eget iaculis purus rhoncus.

Morbi eu lectus arcu. Sed fringilla dui ut magna commodo, a malesuada ante pellentesque. Donec ornare facilisis pellentesque. Nulla vitae fringilla velit. Nunc id tellus nisl. Maecenas pretium elit lectus, nec consectetur nunc vulputate et. Sed facilisis magna nec gravida hendrerit. Sed a cursus nisl, in rhoncus massa. Curabitur ut nibh interdum, tempor risus vel, scelerisque nibh. Mauris quis ipsum sed risus tempor convallis ut a eros.


\begin{savequote}[8cm]
Alles Gescheite ist schon gedacht worden.\\
Man muss nur versuchen, es noch einmal zu denken.

All intelligent thoughts have already been thought;\\
what is necessary is only to try to think them again.
  \qauthor{--- Johann Wolfgang von Goethe \cite{von_goethe_wilhelm_1829}}
\end{savequote}

\chapter{\label{ch:2-litreview}Background}

\minitoc

\section{Introduction}

This document introduction won't serve as a complete primer on \LaTeX.  There are plenty of those online, and googling your questions will often get you answers, especially from \url{http://tex.stackexchange.com}.

Instead, let's talk a little about a few of the features and packages lumped into this template situation.  The \verb|savequote| environment at the beginning of chapters can add some wittiness to your thesis.  If you don't like the quotes, just remove that block.

For when it comes time to do corrections, there are two useful commands here.  First, the \verb|mccorrect| command allows you to highlight a short correction \mccorrect{like this one}.  When the thesis is typeset normally, the correction will just appear as part of the text.  However, when you declare \verb|\correctionstrue| in the main \verb|µthesis.tex| file, that correction will be highlighted in blue.  That might be useful for submitting a post-viva, corrected copy to your examiners so they can quickly verify you've completed the task.

\begin{mccorrection}
For larger chunks, like this paragraph or indeed entire figures, you can use the \verb|mccorrection| environment.  This environment highlights paragraph-sized and larger blocks with the same blue color.
\end{mccorrection}

Read through the \verb|µthesis.tex| file to see the various options for one- and two-sided printing, and turning corrections and draft footer on or off, and the separate option to centre your text on the page (for PDF submission) or offset it (for binding).

\section{Cardiac Imaging}\label{app:imaging}

Within months of Röntgen's discovery of the X-ray in \mccorrect{1895}~\cite{gagliardi_rontgen_1996}, cardiac pathology was being investigated via non-invasive imaging \cite{gagliardi_cardiac_1996}.  Over the intervening years, cardiac imaging modalities and techniques have advanced significantly.  Clinically, cardiac imaging is used for two broad purposes: diagnosis of pathophysiology and guidance of interventional procedures.  These applications impose different requirements on imaging equipment, image acquisition time, computational complexity, spatial and temporal resolution, and tissue discrimination.  The common diagnostic and interventional cardiac imaging techniques in current clinical practice are reviewed below.  An accessible introduction to the physics of medical imaging can be found in \citeauthor{webb_introduction_2002}'s \textit{Introduction to Biomedical Imaging} \cite{webb_introduction_2002}.  A comprehensive overview of the use of imaging in clinical cardiology is presented in \citeauthor{leeson_cardiovascular_2011}'s \textit{Cardiovascular Imaging} \cite{leeson_cardiovascular_2011}.

\subsection{Diagnostic Imaging}
\label{sub:diagnostic}

Beyond the chest X-ray (`plain film'), the key non-invasive imaging modalities in diagnostic cardiology are echocardiography, magnetic resonance imaging, and X-ray computed tomography, which are reviewed below.  Nuclear medicine, including \ac{pet} and \ac{spect}, are not discussed here, as they do not play a role in the chapters to follow.

\begin{figure}
\centering\includegraphics[width=0.7\textwidth]{figures/sample/Gray498.png} 
\caption[Four-chamber illustration of the human heart.]{Four-chamber illustration of the human heart.  Clockwise from upper-left: right atrium, left atrium, left ventricle, right ventricle.}
\label{fig:fourchamber}\end{figure}

\begin{excursion}[Echocardiography]
The use of acoustic waves for medical diagnosis, inspired by naval sonar, was initially developed in the 1940s \cite{gagliardi_ultrasonography_1996}.  By 1954, the first clinically useful cardiac ultrasound -- examining motion of the mitral valve in stenosis -- was reported \cite{edler_ultrasonic_1957}.  These early scans were one-dimensional images (`A-mode'), sometimes repeated to generate a time axis (`M-mode').   The sector-scanning probe was developed in the 1970s \cite{bom_ultrasonic_1971,griffith_sector_1974}, leading to the `B-mode' that a modern cardiologist would recognise as an echocardiogram.
\end{excursion}



%% APPENDICES %% 
% Starts lettered appendices, adds a heading in table of contents, and adds a
%    page that just says "Appendices" to signal the end of your main text.
\startappendices
% Add or remove any appendices you'd like here:
\begin{savequote}[8cm]
\textlatin{Cor animalium, fundamentum eſt vitæ, princeps omnium, Microcoſmi Sol, a quo omnis vegetatio dependet, vigor omnis \& robur emanat.}

The heart of animals is the foundation of their life, the sovereign of everything within them, the sun of their microcosm, that upon which all growth depends, from which all power proceeds.
  \qauthor{--- William Harvey \cite{harvey_exercitatio_1628}}
\end{savequote}

\chapter{\label{app:1-cardiophys}Review of Cardiac Physiology and Electrophysiology}

\minitoc

Appendices are just like chapters.  Their sections and subsections get numbered and included in the table of contents; figures and equations and tables added up, etc.  Lorem ipsum dolor sit amet, consectetur adipiscing elit. Sed et dui sem. Aliquam dictum et ante ut semper. Donec sollicitudin sed quam at aliquet. Sed maximus diam elementum justo auctor, eget volutpat elit eleifend. Curabitur hendrerit ligula in erat feugiat, at rutrum risus suscipit. Pellentesque habitant morbi tristique senectus et netus et malesuada fames ac turpis egestas. Integer risus nulla, facilisis eget lacinia a, pretium mattis metus. Vestibulum aliquam varius ligula nec consectetur. Maecenas ac ipsum odio. Cras ac elit consequat, eleifend ipsum sodales, euismod nunc. Nam vitae tempor enim, sit amet eleifend nisi. Etiam at erat vel neque consequat.

\section{Anatomy}
\label{sec:anatomy}

Lorem ipsum dolor sit amet, consectetur adipiscing elit. Donec accumsan cursus neque. Pellentesque eget tempor turpis, quis malesuada dui. Proin egestas, sapien sit amet feugiat vulputate, nunc nibh mollis nunc, nec auctor turpis purus sed metus. Aenean consequat leo congue volutpat euismod. Vestibulum et vulputate nisl, at ultrices ligula. Cras pulvinar lacinia ipsum at bibendum. In ac augue ut ante mollis molestie in a arcu.

Etiam vitae quam sollicitudin, luctus tortor eu, efficitur nunc. Vestibulum maximus, ante quis consequat sagittis, augue velit luctus odio, in scelerisque arcu magna id diam. Proin et mauris congue magna auctor pretium id sit amet felis. Maecenas sit amet lorem ipsum. Proin a risus diam. Integer tempus eget est condimentum faucibus. Suspendisse sem metus, consequat vel ante eget, porttitor maximus dui. Nunc dapibus tincidunt enim, non aliquam diam vehicula sed. Proin vel felis ut quam porta tempor. Vestibulum elit mi, dictum eget augue non, volutpat imperdiet eros. Praesent ac egestas neque, et vehicula felis.

Pellentesque malesuada volutpat justo, id eleifend leo pharetra at. Pellentesque feugiat rutrum lobortis. Curabitur hendrerit erat porta massa tincidunt rutrum. Donec tincidunt facilisis luctus. Aliquam dapibus sodales consectetur. Suspendisse lacinia, ipsum sit amet elementum fermentum, nulla urna mattis erat, eu porta metus ipsum vel purus. Fusce eget sem nisl. Pellentesque dapibus, urna vitae tristique aliquam, purus leo gravida nunc, id faucibus ipsum magna aliquet ligula. Lorem ipsum dolor sit amet, consectetur adipiscing elit. Proin sem lacus, rutrum eget efficitur sed, aliquam vel augue. Aliquam ut eros vitae sem cursus ultrices ut ornare urna. Nullam tempor porta enim, in pellentesque arcu commodo quis. Interdum et malesuada fames ac ante ipsum primis in faucibus. Curabitur maximus orci purus, ut molestie turpis pellentesque ut.

Donec lacinia tristique ultricies. Proin dignissim risus ut dolor pulvinar mollis. Proin ac turpis vitae nibh finibus ullamcorper viverra quis felis. Mauris pellentesque neque diam, id feugiat diam vestibulum vitae. In suscipit dui eu libero ultrices, et sagittis nunc blandit. Aliquam at aliquet ex. Nullam molestie pulvinar ex vitae interdum. Praesent purus nunc, gravida id est consectetur, convallis elementum nulla. Praesent ex dolor, maximus eu facilisis at, viverra eget nulla. Donec ullamcorper ante nisi. Sed volutpat diam eros. Nullam egestas neque non tortor aliquet, sed pretium velit tincidunt. Aenean condimentum, est ac vestibulum mattis, quam augue congue augue, mattis ultrices nibh libero non ante. Lorem ipsum dolor sit amet, consectetur adipiscing elit.

Aenean volutpat eros tortor, non convallis sapien blandit et. Maecenas faucibus nulla a magna posuere commodo. Nullam laoreet ante a turpis laoreet malesuada. Phasellus in varius sem. Vestibulum sagittis nibh sed tincidunt blandit. Donec aliquam accumsan odio sit amet lacinia. Integer in tellus diam. Vivamus varius massa leo, vitae ullamcorper metus pulvinar sed. Maecenas nec lorem ornare, elementum est quis, gravida massa. Suspendisse volutpat odio ex, ac ultrices leo ultrices vel. Sed sed convallis ipsum. Pellentesque euismod a nulla sed rhoncus. Sed vehicula urna vitae mi aliquet, non sodales lacus ullamcorper. Duis mattis justo turpis, id tempus est tempus eu. Curabitur vitae hendrerit ligula.

Curabitur non pretium enim, in commodo ligula. Etiam commodo eget ligula a lacinia. Vestibulum laoreet ante tellus, vel congue sapien ornare in. Donec venenatis cursus velit vitae pulvinar. Pellentesque habitant morbi tristique senectus et netus et malesuada fames ac turpis egestas. Suspendisse in metus lectus. Pellentesque gravida dolor eget finibus imperdiet. Duis id molestie tortor. Mauris laoreet faucibus facilisis. Aliquam vitae dictum massa, sit amet dignissim lacus.

Fusce eleifend tellus id ex consequat maximus. Donec ultrices ex ut turpis ornare, non molestie mi placerat. Nulla sit amet auctor nunc, sit amet euismod elit. Phasellus risus tellus, condimentum a metus et, venenatis tristique urna. Cras mattis felis eget ipsum fermentum egestas. Ut augue odio, venenatis id convallis vel, congue quis augue. Maecenas sed maximus est, posuere aliquet tortor. Ut condimentum egestas nisi eu porttitor. Ut mi turpis, posuere id lorem vel, elementum tempor arcu.

Morbi nisl arcu, venenatis non metus ac, ullamcorper scelerisque justo. Nulla et accumsan lorem. Mauris aliquet dui sit amet libero aliquet, in ornare metus porttitor. Integer ultricies urna eu consequat ultrices. Maecenas a justo id purus ultricies posuere sed et quam. Cum sociis natoque penatibus et magnis dis parturient montes, nascetur ridiculus mus. Sed eleifend risus quis aliquet gravida. Nullam ac erat porta est bibendum dictum in a dolor. Nam eget turpis viverra, vulputate lectus eget, mattis ligula. Nam at tellus eget dui lobortis sodales et ut augue. In vestibulum diam eget mi cursus, ut tincidunt nulla pellentesque.

Aliquam erat volutpat. Sed ultrices massa id ex mattis bibendum. Nunc augue magna, ornare at aliquet gravida, vehicula sed lorem. Quisque lobortis ipsum eu posuere eleifend. Duis bibendum cursus viverra. Nam venenatis elit leo, vitae feugiat quam aliquet sed. Cras velit est, tempus ac lorem sed, pharetra lobortis ipsum. Donec suscipit gravida interdum. Nunc non finibus est. Nullam turpis elit, tempus non ante.

\section{Mechanical Cycle}

Lorem ipsum dolor sit amet, consectetur adipiscing elit. Aenean tellus est, suscipit sed facilisis quis, malesuada at ipsum. Nam tristique urna quis quam iaculis, et mattis orci pretium. Praesent euismod elit vel metus commodo ultrices. Vestibulum et tincidunt ex, in molestie ex. Donec ullamcorper sollicitudin accumsan. Etiam ac leo turpis. Duis a tortor felis. Nullam sollicitudin eu purus ac hendrerit. Nam hendrerit ligula libero, eget finibus orci bibendum a. Aenean ut ipsum magna.

Ut viverra, sapien sed accumsan blandit, nisi sem tempus tellus, at suscipit magna erat ornare nunc. Proin lacinia, nisi ut rutrum malesuada, nibh quam pellentesque nunc, sit amet consectetur purus felis ac tortor. Suspendisse lacinia ipsum eu sapien pellentesque mattis. Mauris ipsum nunc, placerat non diam vel, efficitur laoreet nunc. Sed lobortis, ipsum eget gravida facilisis, sem nulla viverra mi, in placerat eros sem viverra lacus. Aliquam porta aliquet diam vel commodo. Nulla facilisi. Duis erat libero, lobortis vel hendrerit vitae, sagittis id dui. Nulla pretium eros nec quam tincidunt, vel luctus mi aliquam. Integer imperdiet purus in est tristique venenatis. Ut pellentesque, nunc vitae iaculis ultricies, urna turpis dignissim risus, a laoreet felis magna nec erat.

Quisque sollicitudin faucibus ligula, et egestas nibh dictum sit amet. Proin eu mi a lectus congue pretium eu quis arcu. Suspendisse vehicula libero eu ipsum aliquam, vel elementum nibh mattis. Sed sed sapien vitae turpis tristique pulvinar a ut metus. Etiam semper gravida est, mollis gravida est porta ac. Proin eget tincidunt erat. Maecenas ultrices erat eget purus ultricies, ut lacinia arcu dictum. Nam et nisi sit amet ex congue mattis vel eget lorem. Aliquam erat volutpat. Pellentesque porttitor nibh vitae elementum consectetur. Aenean et est lobortis, congue sapien non, ullamcorper sapien. Ut facilisis sem non dapibus vehicula.

Mauris euismod odio dolor, sit amet gravida mauris placerat et. Curabitur nec dolor non nibh molestie lobortis dignissim non ante. Nullam rutrum lobortis ultrices. Aenean ex erat, elementum sed maximus id, posuere id quam. Proin rutrum ex elit, pretium aliquam risus finibus at. Aenean egestas orci velit, sed aliquet sapien condimentum a. Duis consequat, arcu eu viverra venenatis, dolor lorem gravida lectus, non aliquet nisi sem at augue. Donec laoreet blandit luctus. Aenean vehicula nisl vel faucibus luctus. Sed ut semper velit, vitae laoreet magna. Sed at interdum magna.

Sed iaculis faucibus odio, eu aliquam purus efficitur vel. Cras at nulla ac enim congue varius ut et nulla. Integer blandit mattis augue.

\section{Electrical Cycle}
\label{sec:electcycle}

Lorem ipsum dolor sit amet, consectetur adipiscing elit. In faucibus condimentum rhoncus. Ut dictum nisl id risus gravida lobortis. Sed vehicula mollis tellus ut varius. Fusce eget egestas dui, et commodo dui. Proin sollicitudin interdum tempus. Nullam in elit a enim fringilla bibendum. Vestibulum sodales pellentesque condimentum. Nulla facilisi. Nunc et dolor in nulla eleifend dictum at vel ligula. Aliquam ut velit non elit ullamcorper porta ac et ex. Fusce ornare magna non nunc vestibulum, eget molestie quam dictum. In interdum aliquam odio, in posuere tellus convallis quis. Curabitur non diam elit. Proin vulputate orci diam, a tincidunt ante luctus eu. Ut a viverra ligula. Curabitur pulvinar tempus tellus eget suscipit.

Aliquam posuere massa at ante dapibus congue. Curabitur ullamcorper tortor eget consectetur aliquet. Mauris tempor magna id mauris fringilla, a varius erat blandit. Nam eleifend ullamcorper placerat. Phasellus augue tortor, volutpat bibendum lorem nec, fringilla volutpat nisl. Mauris cursus urna metus, vel eleifend orci iaculis ut. Sed sit amet scelerisque massa, quis consequat dui. Donec semper sem dui, ac placerat velit egestas vel. Nulla facilisi. Quisque tellus eros, sagittis malesuada augue ut, faucibus dictum nulla. Vestibulum non dapibus erat, ut consequat libero. Ut turpis mi, dapibus commodo libero lobortis, maximus vestibulum lectus. Vestibulum sit amet sapien dapibus, tincidunt leo in, suscipit arcu. Sed in erat bibendum, laoreet eros eu, pellentesque justo. Nulla sodales purus neque, eget maximus ipsum consequat at. Maecenas a nisl sagittis, tempus ipsum sed, dictum mauris.

Suspendisse posuere odio lacus, at auctor tortor vehicula sed. Phasellus suscipit ornare enim vitae placerat. Sed viverra purus vel sapien tempor, quis iaculis erat laoreet. Aenean vel nunc vestibulum, ornare nunc ac, mollis urna. Aenean ultrices felis ipsum, ac semper est ullamcorper in. Donec in justo varius, egestas tortor ut, venenatis augue. Duis mattis, ligula quis lacinia fringilla, tellus neque accumsan ipsum, vitae tempor metus elit vel nibh. Curabitur porttitor urna nec sapien tempor, et porttitor velit malesuada.

Suspendisse aliquam nisl quis placerat vulputate. Proin dapibus ipsum ac ante sagittis, volutpat auctor sem dapibus. Nam in facilisis odio. Integer ante mauris, eleifend et pulvinar in, venenatis quis ligula. Phasellus posuere sollicitudin tortor eget euismod. Maecenas mollis tortor eget justo vulputate sagittis. Etiam hendrerit massa quis ex molestie sodales. Quisque facilisis erat lacus, id convallis sem suscipit bibendum. Integer dui urna, pharetra sed porta sed, bibendum ut odio. Donec placerat at lectus egestas consequat. Sed id rhoncus est, vitae vulputate sapien. Fusce tempus quam lorem, id ornare turpis sodales sed. Integer aliquet urna eget condimentum consequat. Vestibulum quis dui vel ligula posuere luctus id nec turpis.

Nam vitae placerat lacus. Mauris scelerisque interdum volutpat. Nunc aliquet tristique enim, sit amet molestie felis ullamcorper vitae. Nullam sollicitudin orci orci, in condimentum tellus consectetur in. Nam id justo justo. Fusce eget finibus est. Proin id tortor nec quam cursus vehicula. Aliquam ultrices eros eros, a tincidunt elit eleifend auctor.

Nullam consectetur dapibus ligula sit amet efficitur. Nunc non posuere sapien. Vivamus dui nisl, aliquam id ipsum non, pulvinar ornare neque. Nunc rhoncus pretium congue. Fusce id laoreet enim. Cras sed massa in eros bibendum auctor in nec sem. Nam commodo, velit id porta consequat, mi arcu gravida lorem, ut aliquam elit ante quis dui. Quisque in massa sed nibh blandit dictum.

Vestibulum molestie consectetur porttitor. Donec tincidunt vel orci at pharetra. Nullam id felis sit amet nulla tempus lacinia. Integer egestas ullamcorper massa, ut ultricies diam congue sit amet. Cras sit amet velit at nibh vehicula finibus a et lorem. Cras odio metus, venenatis ut ultrices non, ornare ac orci. Morbi et nulla dui. Mauris dictum molestie nibh, eu efficitur lorem accumsan quis.

\section{Cellular Electromechanical Coupling}
\label{sec:electromech}

Lorem ipsum dolor sit amet, consectetur adipiscing elit. Nullam vitae consectetur metus, ac maximus ex. Quisque vitae ex eu lectus ultricies consequat vel non lorem. Etiam odio ipsum, tempus ut lobortis in, molestie ac leo. Vivamus mollis feugiat bibendum. Vestibulum eget venenatis quam. Aenean faucibus, massa sed ullamcorper porta, arcu nunc iaculis velit, quis consectetur purus neque placerat nibh. Vestibulum elit nunc, dignissim vulputate venenatis et, sodales non massa. Proin leo ligula, vehicula eu aliquam varius, posuere a dolor. Donec iaculis auctor neque, sit amet gravida libero porta vel. Vivamus consequat elementum lacus, at bibendum mauris egestas nec. Fusce fermentum diam eu dolor ornare, vitae vestibulum leo interdum. Morbi luctus libero quis dictum laoreet. Etiam semper porta ante, vel ullamcorper enim sodales quis.

Nullam eu nisi faucibus, fermentum ex auctor, tempor arcu. Phasellus condimentum erat mi, condimentum malesuada ligula congue venenatis. Nullam gravida imperdiet urna quis cursus. Ut tempus nec purus eget posuere. Cras non nulla sit amet justo aliquet pellentesque nec sed eros. Nam aliquam nisl urna, in placerat magna gravida venenatis. Donec interdum vel magna ullamcorper molestie. Nunc felis neque, rhoncus fringilla faucibus sit amet, ultrices sed magna. Maecenas malesuada hendrerit diam in ultrices. Nam libero urna, volutpat ut auctor eget, interdum sed odio. Vestibulum suscipit mauris nec augue ornare, ut eleifend nulla gravida. Proin imperdiet, mauris quis consectetur porta, leo dui convallis leo, id lobortis massa diam eu libero. Aenean hendrerit vel ante aliquam venenatis. Pellentesque bibendum pretium odio, ut sagittis lectus feugiat a. Donec porttitor vulputate lacus.

Nunc volutpat efficitur lacus in aliquet. Nullam non iaculis diam, at ultrices diam. Proin vehicula vulputate cursus. Morbi tempus sapien id urna lobortis interdum. Maecenas elementum sagittis elementum. Donec at sodales velit, a posuere tortor. Nulla id hendrerit tortor. Sed semper velit in magna sagittis pulvinar. Nulla nec arcu molestie, ultricies sapien sit amet, sollicitudin nisi. Donec nisi massa, suscipit ut dignissim quis, lacinia id leo.

Suspendisse ut mi metus. Morbi tincidunt ligula in porttitor consectetur. Integer eu urna urna. Suspendisse potenti. Mauris sit amet felis eu diam auctor ullamcorper. Morbi in porta nisi. Nam ante tortor, venenatis vitae tempor sed, sagittis vitae velit. In semper orci sit amet nisi ullamcorper varius. Aenean dignissim ultrices imperdiet. Maecenas lacinia enim id neque porttitor iaculis. Curabitur laoreet ante ut urna dignissim, id sollicitudin metus consectetur. Aenean massa ipsum, auctor vel ante vel, blandit dignissim libero. Fusce interdum ac magna et interdum.



%%%%% REFERENCES

% JEM: Quote for the top of references (just like a chapter quote if you're using them).  Comment to skip.
\begin{savequote}[8cm]
The first kind of intellectual and artistic personality belongs to the hedgehogs, the second to the foxes \dots
  \qauthor{--- Sir Isaiah Berlin \cite{berlin_hedgehog_2013}}
\end{savequote}

\setlength{\baselineskip}{0pt} % JEM: Single-space References

{\renewcommand*\MakeUppercase[1]{#1}%
\newrefcontext[sorting=none]%
\printbibliography[heading=bibintoc,title={\bibtitle}]}


\end{document}
