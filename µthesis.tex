%%%%%%%%%%%%%%%%%%%%%%%%%%%%%%%%%%%%%%%%%%%%%%%%%%%%%%%%%%%%%%%
%% OXFORD THESIS TEMPLATE

% Use this template to produce a standard thesis that meets the Oxford University requirements for DPhil submission
%
% Originally by Keith A. Gillow (gillow@maths.ox.ac.uk), 1997
% Modified by Sam Evans (sam@samuelevansresearch.org), 2007
% Modified by John McManigle (john@oxfordechoes.com), 2015
%
% This version Copyright (c) 2015-2023 John McManigle
%
% Broad permissions are granted to use, modify, and distribute this software
% as specified in the MIT License included in this distribution's LICENSE file.
%

% I've (John) tried to comment this file extensively, so read through it to see how to use the various options.  Remember
% that in LaTeX, any line starting with a % is NOT executed.  Several places below, you have a choice of which line to use
% out of multiple options (eg draft vs final, for PDF vs for binding, etc.)  When you pick one, add a % to the beginning of
% the lines you don't want.


%%%%% CHOOSE PAGE LAYOUT
% The most common choices should be below.  You can also do other things, like replacing "a4paper" with "letterpaper", etc.

% This one will format for two-sided binding (ie left and right pages have mirror margins; blank pages inserted where needed):
\documentclass[a4paper,twoside]{µthesis}
% This one will format for one-sided binding (ie left margin > right margin; no extra blank pages):
%\documentclass[a4paper]{µthesis}
% This one will format for PDF output (ie equal margins, no extra blank pages):
%\documentclass[a4paper,nobind]{µthesis} 

% Use Noto fonts - comment out to use default LaTeX fonts
\usepackage{fontspec}
\usepackage{noto-mono}
\usepackage{notomath}

% use \cref{sec} to reference a section - produces "Section 1"
\usepackage[capitalize,noabbrev,nameinlink]{cleveref}


%%%%% SELECT YOUR DRAFT OPTIONS
% Three options going on here; use in any combination.  But remember to turn the first two off before
% generating a PDF to send to the printer!

% This adds a "DRAFT" footer to every normal page.  (The first page of each chapter is not a "normal" page.)
\fancyfoot[C]{\emph{DRAFT Printed on \today}}  

% This highlights (in blue) corrections marked with (for words) \mccorrect{blah} or (for whole
% paragraphs) \begin{mccorrection} . . . \end{mccorrection}.  This can be useful for sending a PDF of
% your corrected thesis to your examiners for review.  Turn it off, and the blue disappears.
\correctionstrue


%%%%% BIBLIOGRAPHY SETUP

\usepackage[style=ieee, sortcites=true, backend=biber, doi=true, isbn=false,
            maxnames=1, minnames=1, maxbibnames=99, citestyle=numeric-comp]{biblatex}
\newcommand*{\bibtitle}{References}

% Environment for the Puplications page
\defbibenvironment{nolabelspaced}
  {\list{}{\leftmargin=0pt
           \itemindent=0pt
           \labelwidth=0pt
           \labelsep=0pt
           \setlength{\itemsep}{4ex} % space between items
           \setlength{\parsep}{0pt}
           }
   \renewcommand*{\makelabel}[1]{}}%
  {\endlist}
  {\item}

% You can use the following snippet to select your publications, just replace "Patton" with your name.
% Alternatively, you can set "keywords={own}" for bib entries your dissertation is based on.
\DeclareSourcemap{
  \maps[datatype=bibtex]{
    \map{
      \step[fieldsource=author, match=Patton, final]
      \step[fieldset=keywords, fieldvalue=own]
    }
  }
}

% This makes the bibliography use a slightly smaller font.
\renewcommand*{\bibfont}{\small}

% Suppress warnings...
\DeclareLanguageMapping{latin}{english}
\BiblatexSplitbibDefernumbersWarningOff

% Change this to the name of your .bib file (usually exported from a citation manager like Zotero or EndNote).
\addbibresource{references.bib}


% Uncomment this if you want equation numbers per section (2.3.12), instead of per chapter (2.18):
%\numberwithin{equation}{subsection}




%%%%% THESIS / TITLE PAGE INFORMATION
\title{Suitably impressive thesis title}
\author{Your Name}
\department{Informatik}
% Your full degree name.
\degree{Doktors (Dr.)}
% Your examiners/supervisors.
\firstexaminer{Prof. A}
\secondexaminer{Prof. B}
% You don't need these dates for your initial submission, they must be set for the final submission.
% \submissiondate{01.01.2026}
% \acceptancedate{01.02.2026}
% \examinationdate{01.03.2026}


%%%%% YOUR OWN PERSONAL MACROS
% This is a good place to dump your own LaTeX macros as they come up.

% To make text superscripts shortcuts
	\renewcommand{\th}{\textsuperscript{th}} % ex: I won 4\th place
	\newcommand{\nd}{\textsuperscript{nd}}
	\renewcommand{\st}{\textsuperscript{st}}
	\newcommand{\rd}{\textsuperscript{rd}}

%%%%% BOXED ENVIRONMENTS
% Some suggestions for boxed environments, see a usage example in Chapter 2.
\newboxenv{excursion}{Excursion}{unibwgraylight}
\newboxenv{definition}{Definition}{unibwbluelight}
\newboxenv{observation}{Observation}{unibworangelight}

\DeclareAcronymWithTooltip{pet}{PET}{positron emission tomography}
\DeclareAcronymWithTooltip{spect}{SPECT}{single-photon emission computed tomography}

% or use \DeclareAcronym if you don't want tooltips

%%%%% THE ACTUAL DOCUMENT STARTS HERE
\begin{document}



%%%%% CHOOSE YOUR LINE SPACING HERE
\setlength{\textbaselineskip}{15pt plus2pt minus1pt}

% You can set the spacing here for the roman-numbered pages (acknowledgements, table of contents, etc.)
\setlength{\frontmatterbaselineskip}{15pt plus2pt minus1pt}

% Leave this line alone; it gets things started for the real document.
\setlength{\baselineskip}{\textbaselineskip}


%%%%% CHOOSE YOUR SECTION NUMBERING DEPTH HERE
% You have two choices.  First, how far down are sections numbered?  (Below that, they're named but
% don't get numbers.)  Second, what level of section appears in the table of contents?  These don't have
% to match: you can have numbered sections that don't show up in the ToC, or unnumbered sections that
% do.  Throughout, 0 = chapter; 1 = section; 2 = subsection; 3 = subsubsection, 4 = paragraph...

% The level that gets a number:
\setcounter{secnumdepth}{2}
% The level that shows up in the ToC:
\setcounter{tocdepth}{2}


% JEM: Pages are roman numbered from here, though page numbers are invisible until ToC.  This is in
% keeping with most typesetting conventions.
\begin{romanpages}

% JEM: By default, this template uses the traditional Oxford "Belt Crest". Un-comment the following
% line to use the newer, "Blue Square" logo:
% \renewcommand{\crest}{{\includegraphics[width=4.2cm, height=4.2cm]{figures/newlogo.pdf}}}

% Title page is created here
\maketitle

%%%%% DEDICATION -- If you'd like one, un-comment the following.
%\begin{dedication}
%This thesis is dedicated to\\
%someone\\
%for some special reason\\
%\end{dedication}

%%%%% ACKNOWLEDGEMENTS -- Nothing to do here except comment out if you don't want it.
\begin{acknowledgements}
 	\input{preface/acknowledgements}
\end{acknowledgements}

%%%%% ABSTRACT -- Nothing to do here except comment out if you don't want it.
\renewcommand{\abstractname}{Kurzfassung}
\begin{abstract}
    \input{preface/kurzfassung}
\end{abstract}
\cleardoublepage

\renewcommand{\abstractname}{Abstract}
\begin{abstract}
    \input{preface/abstract}
\end{abstract}
\cleardoublepage

%%%%% MINI TABLES
% This lays the groundwork for per-chapter, mini tables of contents.  Comment the following line
% (and remove \minitoc from the chapter files) if you don't want this.  Un-comment either of the
% next two lines if you want a per-chapter list of figures or tables.
\dominitoc % include a mini table of contents
%\dominilof  % include a mini list of figures
%\dominilot  % include a mini list of tables

% This aligns the bottom of the text of each page.  It generally makes things look better.
\flushbottom

% This is where the whole-document ToC appears:
\tableofcontents

\listoffigures
	\mtcaddchapter
% \mtcaddchapter is needed when adding a non-chapter (but chapter-like) entity to avoid confusing minitoc

% Uncomment to generate a list of tables:
\listoftables
	\mtcaddchapter

%%%%% LIST OF ABBREVIATIONS
% This example includes a list of abbreviations.  Look at preface/abbreviations.tex to see how that file is formatted.
\printacronyms[name=List of Abbreviations]
\mtcaddchapter[List of Abbreviations]

%%%%% LIST OF PUBLICATIONS
% Generates a list of all publications in references.bib with the keyword "own".
\chapter*{Publications}
\mtcaddchapter[Publications] 
This dissertation is based on the following peer-reviewed publications, ordered by recency:
\vspace{2ex}
\newrefcontext[sorting=ddnt] % most recent first
\printbibliography[keyword=own, heading=none, env=nolabelspaced]

% The Roman pages, like the Roman Empire, must come to its inevitable close.
\end{romanpages}


%%%%% CHAPTERS
% Add or remove any chapters you'd like here, by file name (excluding '.tex'):
\flushbottom
\include{chapters/ch1-intro}
\include{chapters/ch2-litreview}


%% APPENDICES %% 
% Starts lettered appendices, adds a heading in table of contents, and adds a
%    page that just says "Appendices" to signal the end of your main text.
\startappendices
% Add or remove any appendices you'd like here:
\include{appendices/appendix-1}


%%%%% REFERENCES

% JEM: Quote for the top of references (just like a chapter quote if you're using them).  Comment to skip.
\begin{savequote}[8cm]
The first kind of intellectual and artistic personality belongs to the hedgehogs, the second to the foxes \dots
  \qauthor{--- Sir Isaiah Berlin \cite{berlin_hedgehog_2013}}
\end{savequote}

\setlength{\baselineskip}{0pt} % JEM: Single-space References

{\renewcommand*\MakeUppercase[1]{#1}%
\newrefcontext[sorting=none]%
\printbibliography[heading=bibintoc,title={\bibtitle}]}


\end{document}
